% Options for packages loaded elsewhere
\PassOptionsToPackage{unicode}{hyperref}
\PassOptionsToPackage{hyphens}{url}
%
\documentclass[
]{book}
\usepackage{amsmath,amssymb}
\usepackage{iftex}
\ifPDFTeX
  \usepackage[T1]{fontenc}
  \usepackage[utf8]{inputenc}
  \usepackage{textcomp} % provide euro and other symbols
\else % if luatex or xetex
  \usepackage{unicode-math} % this also loads fontspec
  \defaultfontfeatures{Scale=MatchLowercase}
  \defaultfontfeatures[\rmfamily]{Ligatures=TeX,Scale=1}
\fi
\usepackage{lmodern}
\ifPDFTeX\else
  % xetex/luatex font selection
\fi
% Use upquote if available, for straight quotes in verbatim environments
\IfFileExists{upquote.sty}{\usepackage{upquote}}{}
\IfFileExists{microtype.sty}{% use microtype if available
  \usepackage[]{microtype}
  \UseMicrotypeSet[protrusion]{basicmath} % disable protrusion for tt fonts
}{}
\makeatletter
\@ifundefined{KOMAClassName}{% if non-KOMA class
  \IfFileExists{parskip.sty}{%
    \usepackage{parskip}
  }{% else
    \setlength{\parindent}{0pt}
    \setlength{\parskip}{6pt plus 2pt minus 1pt}}
}{% if KOMA class
  \KOMAoptions{parskip=half}}
\makeatother
\usepackage{xcolor}
\usepackage{color}
\usepackage{fancyvrb}
\newcommand{\VerbBar}{|}
\newcommand{\VERB}{\Verb[commandchars=\\\{\}]}
\DefineVerbatimEnvironment{Highlighting}{Verbatim}{commandchars=\\\{\}}
% Add ',fontsize=\small' for more characters per line
\usepackage{framed}
\definecolor{shadecolor}{RGB}{248,248,248}
\newenvironment{Shaded}{\begin{snugshade}}{\end{snugshade}}
\newcommand{\AlertTok}[1]{\textcolor[rgb]{0.94,0.16,0.16}{#1}}
\newcommand{\AnnotationTok}[1]{\textcolor[rgb]{0.56,0.35,0.01}{\textbf{\textit{#1}}}}
\newcommand{\AttributeTok}[1]{\textcolor[rgb]{0.13,0.29,0.53}{#1}}
\newcommand{\BaseNTok}[1]{\textcolor[rgb]{0.00,0.00,0.81}{#1}}
\newcommand{\BuiltInTok}[1]{#1}
\newcommand{\CharTok}[1]{\textcolor[rgb]{0.31,0.60,0.02}{#1}}
\newcommand{\CommentTok}[1]{\textcolor[rgb]{0.56,0.35,0.01}{\textit{#1}}}
\newcommand{\CommentVarTok}[1]{\textcolor[rgb]{0.56,0.35,0.01}{\textbf{\textit{#1}}}}
\newcommand{\ConstantTok}[1]{\textcolor[rgb]{0.56,0.35,0.01}{#1}}
\newcommand{\ControlFlowTok}[1]{\textcolor[rgb]{0.13,0.29,0.53}{\textbf{#1}}}
\newcommand{\DataTypeTok}[1]{\textcolor[rgb]{0.13,0.29,0.53}{#1}}
\newcommand{\DecValTok}[1]{\textcolor[rgb]{0.00,0.00,0.81}{#1}}
\newcommand{\DocumentationTok}[1]{\textcolor[rgb]{0.56,0.35,0.01}{\textbf{\textit{#1}}}}
\newcommand{\ErrorTok}[1]{\textcolor[rgb]{0.64,0.00,0.00}{\textbf{#1}}}
\newcommand{\ExtensionTok}[1]{#1}
\newcommand{\FloatTok}[1]{\textcolor[rgb]{0.00,0.00,0.81}{#1}}
\newcommand{\FunctionTok}[1]{\textcolor[rgb]{0.13,0.29,0.53}{\textbf{#1}}}
\newcommand{\ImportTok}[1]{#1}
\newcommand{\InformationTok}[1]{\textcolor[rgb]{0.56,0.35,0.01}{\textbf{\textit{#1}}}}
\newcommand{\KeywordTok}[1]{\textcolor[rgb]{0.13,0.29,0.53}{\textbf{#1}}}
\newcommand{\NormalTok}[1]{#1}
\newcommand{\OperatorTok}[1]{\textcolor[rgb]{0.81,0.36,0.00}{\textbf{#1}}}
\newcommand{\OtherTok}[1]{\textcolor[rgb]{0.56,0.35,0.01}{#1}}
\newcommand{\PreprocessorTok}[1]{\textcolor[rgb]{0.56,0.35,0.01}{\textit{#1}}}
\newcommand{\RegionMarkerTok}[1]{#1}
\newcommand{\SpecialCharTok}[1]{\textcolor[rgb]{0.81,0.36,0.00}{\textbf{#1}}}
\newcommand{\SpecialStringTok}[1]{\textcolor[rgb]{0.31,0.60,0.02}{#1}}
\newcommand{\StringTok}[1]{\textcolor[rgb]{0.31,0.60,0.02}{#1}}
\newcommand{\VariableTok}[1]{\textcolor[rgb]{0.00,0.00,0.00}{#1}}
\newcommand{\VerbatimStringTok}[1]{\textcolor[rgb]{0.31,0.60,0.02}{#1}}
\newcommand{\WarningTok}[1]{\textcolor[rgb]{0.56,0.35,0.01}{\textbf{\textit{#1}}}}
\usepackage{longtable,booktabs,array}
\usepackage{calc} % for calculating minipage widths
% Correct order of tables after \paragraph or \subparagraph
\usepackage{etoolbox}
\makeatletter
\patchcmd\longtable{\par}{\if@noskipsec\mbox{}\fi\par}{}{}
\makeatother
% Allow footnotes in longtable head/foot
\IfFileExists{footnotehyper.sty}{\usepackage{footnotehyper}}{\usepackage{footnote}}
\makesavenoteenv{longtable}
\usepackage{graphicx}
\makeatletter
\def\maxwidth{\ifdim\Gin@nat@width>\linewidth\linewidth\else\Gin@nat@width\fi}
\def\maxheight{\ifdim\Gin@nat@height>\textheight\textheight\else\Gin@nat@height\fi}
\makeatother
% Scale images if necessary, so that they will not overflow the page
% margins by default, and it is still possible to overwrite the defaults
% using explicit options in \includegraphics[width, height, ...]{}
\setkeys{Gin}{width=\maxwidth,height=\maxheight,keepaspectratio}
% Set default figure placement to htbp
\makeatletter
\def\fps@figure{htbp}
\makeatother
\setlength{\emergencystretch}{3em} % prevent overfull lines
\providecommand{\tightlist}{%
  \setlength{\itemsep}{0pt}\setlength{\parskip}{0pt}}
\setcounter{secnumdepth}{5}
\newlength{\cslhangindent}
\setlength{\cslhangindent}{1.5em}
\newlength{\csllabelwidth}
\setlength{\csllabelwidth}{3em}
\newlength{\cslentryspacingunit} % times entry-spacing
\setlength{\cslentryspacingunit}{\parskip}
\newenvironment{CSLReferences}[2] % #1 hanging-ident, #2 entry spacing
 {% don't indent paragraphs
  \setlength{\parindent}{0pt}
  % turn on hanging indent if param 1 is 1
  \ifodd #1
  \let\oldpar\par
  \def\par{\hangindent=\cslhangindent\oldpar}
  \fi
  % set entry spacing
  \setlength{\parskip}{#2\cslentryspacingunit}
 }%
 {}
\usepackage{calc}
\newcommand{\CSLBlock}[1]{#1\hfill\break}
\newcommand{\CSLLeftMargin}[1]{\parbox[t]{\csllabelwidth}{#1}}
\newcommand{\CSLRightInline}[1]{\parbox[t]{\linewidth - \csllabelwidth}{#1}\break}
\newcommand{\CSLIndent}[1]{\hspace{\cslhangindent}#1}
\ifLuaTeX
  \usepackage{selnolig}  % disable illegal ligatures
\fi
\IfFileExists{bookmark.sty}{\usepackage{bookmark}}{\usepackage{hyperref}}
\IfFileExists{xurl.sty}{\usepackage{xurl}}{} % add URL line breaks if available
\urlstyle{same}
\hypersetup{
  pdftitle={Fluent Graphics},
  pdfauthor={Adam Bartonicek},
  hidelinks,
  pdfcreator={LaTeX via pandoc}}

\title{Fluent Graphics}
\author{Adam Bartonicek}
\date{}

\begin{document}
\maketitle

{
\setcounter{tocdepth}{1}
\tableofcontents
}
\hypertarget{abstract}{%
\chapter{Abstract}\label{abstract}}

Placeholder

\hypertarget{introduction}{%
\chapter{Introduction}\label{introduction}}

Placeholder

\hypertarget{what-even-is-interactive-data-visualization}{%
\section{What Even is Interactive Data Visualization?}\label{what-even-is-interactive-data-visualization}}

\hypertarget{brief-history-of-interactive-data-visualization}{%
\subsection{Brief History of Interactive Data Visualization}\label{brief-history-of-interactive-data-visualization}}

\hypertarget{rise-of-interactive-data-visualization-by-statisticians-for-statisticians}{%
\subsubsection{Rise of Interactive Data Visualization: By Statisticians for Statisticians}\label{rise-of-interactive-data-visualization-by-statisticians-for-statisticians}}

\hypertarget{interactive-data-visualization-and-the-web-interactivity-for-everyone}{%
\subsubsection{Interactive Data Visualization and the Web: Interactivity for Everyone}\label{interactive-data-visualization-and-the-web-interactivity-for-everyone}}

\hypertarget{interactive-vs.-interacting-with}{%
\subsection{Interactive vs.~Interacting With}\label{interactive-vs.-interacting-with}}

\hypertarget{the-bar-for-interactivity}{%
\subsection{The Bar for Interactivity}\label{the-bar-for-interactivity}}

\hypertarget{interactivity-and-implementation}{%
\subsection{Interactivity and Implementation}\label{interactivity-and-implementation}}

\hypertarget{working-definition-of-interactivity}{%
\subsection{Working Definition of Interactivity}\label{working-definition-of-interactivity}}

\hypertarget{common-interactive-features}{%
\subsection{Common Interactive Features}\label{common-interactive-features}}

\hypertarget{mathematical-theory}{%
\section{Mathematical Theory}\label{mathematical-theory}}

\hypertarget{past-application-of-category-theory-to-data-visualization}{%
\subsection{Past Application of Category Theory to Data Visualization}\label{past-application-of-category-theory-to-data-visualization}}

\hypertarget{relations}{%
\subsection{Relations}\label{relations}}

\hypertarget{functions}{%
\subsection{Functions}\label{functions}}

\hypertarget{partitions}{%
\subsection{Partitions}\label{partitions}}

\hypertarget{preorders}{%
\subsection{Preorders}\label{preorders}}

\hypertarget{monoids}{%
\subsection{Monoids}\label{monoids}}

\hypertarget{components-of-a-data-visualization-system}{%
\section{Components of a Data Visualization System}\label{components-of-a-data-visualization-system}}

\hypertarget{scales}{%
\subsection{Scales}\label{scales}}

\hypertarget{nominal-scales}{%
\paragraph{Nominal scales}\label{nominal-scales}}

\hypertarget{ordinal-scales}{%
\paragraph{Ordinal scales}\label{ordinal-scales}}

\hypertarget{interval-scales}{%
\paragraph{Interval scales}\label{interval-scales}}

\hypertarget{ratio-scales}{%
\subparagraph{Ratio scales}\label{ratio-scales}}

\hypertarget{criticism-of-on-the-theory-of-scales-of-measurement}{%
\paragraph{Criticism of On the Theory of Scales of Measurement}\label{criticism-of-on-the-theory-of-scales-of-measurement}}

\hypertarget{design}{%
\chapter{Design}\label{design}}

Placeholder

\hypertarget{user-profile}{%
\subsection{User Profile}\label{user-profile}}

\hypertarget{implementation}{%
\section{Implementation}\label{implementation}}

\hypertarget{factors}{%
\subsection{Factors}\label{factors}}

\hypertarget{product-factors}{%
\subsubsection{Product factors}\label{product-factors}}

\hypertarget{reducers}{%
\subsection{Reducers}\label{reducers}}

\hypertarget{scales-1}{%
\subsection{Scales}\label{scales-1}}

\hypertarget{expanses}{%
\subsection{Expanses}\label{expanses}}

\hypertarget{zero-and-one}{%
\subsubsection{Zero and One}\label{zero-and-one}}

\hypertarget{expanse-interface}{%
\subsubsection{Expanse Interface}\label{expanse-interface}}

\hypertarget{continuous-expanses}{%
\subsubsection{Continuous Expanses}\label{continuous-expanses}}

\hypertarget{linearity}{%
\paragraph{Linearity}\label{linearity}}

\hypertarget{transformations}{%
\paragraph{Transformations}\label{transformations}}

\hypertarget{section}{%
\section{}\label{section}}

\hypertarget{glossary}{%
\chapter{Glossary}\label{glossary}}

\hypertarget{JSON}{%
\section{JSON}\label{JSON}}

Short for ``JavaScript Object Notation'', JSON is a flexible data format based on the JavaScript object type (\protect\hyperlink{ref-ecma2024}{Ecma International 2024}; see also e.g. \protect\hyperlink{ref-bourhis2017}{Bourhis et al. 2017}; \protect\hyperlink{ref-pezoa2016}{Pezoa et al. 2016}). On the highest level, a JSON is an dictionary (also known as object, struct, hash-table, or list in other languages) containing key-value pairs, where the keys are strings and the values can be any of the following types: string, number, boolean, null (an undefined/missing value), an array (which can contain any other valid JSON values), or another JSON object.

For example, the following is a valid JSON:

\begin{verbatim}
{
  "name": "Adam",
  "age": 30,
  "friends": [{ "name": "Sam", "age": 30 }, { "name": "Franta", "age": 26}],
  "can drive": true,
  "problems": null
}
\end{verbatim}

It is important to keep in mind that a JSON is more limited than a true JavaScript object type as implemented in the browser and various JavaScript runtimes. JavaScript objects are very flexible and can have non-string keys (numbers or symbols) and contain a wider variety of values, most notably functions/methods (and, of course, both the keys and values can be changed during runtime). In contrast, JSON is a ``simple'' static data format designed for declaring and transporting data. Specifically, JSON is often used as the medium for sending data to and from Web APIs (\protect\hyperlink{ref-bourhis2017}{Bourhis et al. 2017}; \protect\hyperlink{ref-pezoa2016}{Pezoa et al. 2016}) as well as for configuration documents.

The main advantages of JSON are that it is a fairly simple, yet very flexible and human-readable format. Due to its recursive nature (JSON arrays and objects can contain other JSON arrays and objects), it can be used to express a wide variety of hierarchical data which would be inefficient to express in ``flat'' data formats such as CSV. However, this flexibility also comes with some disadvantages. The recursive nature of the format makes parsing JSON files inherently more time-intensive, and, since the values in a JSON can be of any type (as long as it is a valid JSON type), it is often necessary to validate inputs (\protect\hyperlink{ref-pezoa2016}{Pezoa et al. 2016}).

\hypertarget{svg}{%
\section{SVG}\label{svg}}

Short for ``Scalable Vector Graphics'', SVG is a markup language for defining vector graphics (\protect\hyperlink{ref-mdn2024b}{MDN 2024}). Based on XML, SVG graphics can be specified via a hierarchy of elements enclosed by tags, which may be given attributes.

For example, the following is a valid SVG:

\begin{Shaded}
\begin{Highlighting}[]
\KeywordTok{\textless{}svg} \ErrorTok{width}\OtherTok{=}\StringTok{"400"} \ErrorTok{height}\OtherTok{=}\StringTok{"400"}\KeywordTok{\textgreater{}}
  \KeywordTok{\textless{}circle} \ErrorTok{cx}\OtherTok{=}\StringTok{"200"} \ErrorTok{cy}\OtherTok{=}\StringTok{"200"} \ErrorTok{r}\OtherTok{=}\StringTok{"50"} \ErrorTok{fill}\OtherTok{=}\StringTok{"skyblue"}\KeywordTok{\textgreater{}\textless{}/circle\textgreater{}}
  \KeywordTok{\textless{}rect} \ErrorTok{x}\OtherTok{=}\StringTok{"150"} \ErrorTok{y}\OtherTok{=}\StringTok{"150"} \ErrorTok{width}\OtherTok{=}\StringTok{"50"} \ErrorTok{height}\OtherTok{=}\StringTok{"50"} \ErrorTok{fill}\OtherTok{=}\StringTok{"firebrick"}\KeywordTok{\textgreater{}\textless{}/rect\textgreater{}}
\KeywordTok{\textless{}/svg\textgreater{}}
\end{Highlighting}
\end{Shaded}

\hypertarget{references}{%
\chapter{References}\label{references}}

\hypertarget{refs}{}
\begin{CSLReferences}{1}{0}
\leavevmode\vadjust pre{\hypertarget{ref-bourhis2017}{}}%
Bourhis, Pierre, Juan L Reutter, Fernando Suárez, and Domagoj Vrgoč. 2017. {``JSON: Data Model, Query Languages and Schema Specification.''} In \emph{Proceedings of the 36th ACM SIGMOD-SIGACT-SIGAI Symposium on Principles of Database Systems}, 123--35.

\leavevmode\vadjust pre{\hypertarget{ref-ecma2024}{}}%
Ecma International. 2024. {``JSON.''} \url{https://www.json.org/json-en.html}.

\leavevmode\vadjust pre{\hypertarget{ref-mdn2024b}{}}%
MDN. 2024. {``SVG: Scalable Vector Graphics {\(\vert\)} MDN.''} \emph{MDN Web Docs}. \url{https://developer.mozilla.org/en-US/docs/Web/SVG}.

\leavevmode\vadjust pre{\hypertarget{ref-pezoa2016}{}}%
Pezoa, Felipe, Juan L Reutter, Fernando Suarez, Martı́n Ugarte, and Domagoj Vrgoč. 2016. {``Foundations of JSON Schema.''} In \emph{Proceedings of the 25th International Conference on World Wide Web}, 263--73.

\end{CSLReferences}

\end{document}
